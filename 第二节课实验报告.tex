%! Author = 32186
%! Date = 2024/9/11

% Preamble
\documentclass[16pt]{lzc}

\title{Shell 和 vim 学习实验报告}
\author{刘志才}
\course{系统开发工具基础}
\date{8月30日}

% Document
\begin{document}

    \maketitlepage
    \tableofcontents
    \newpage
    \setcounter{page}{1}


    \section{Shell}\label{sec:shell}

        使用 cd 命令可以进入文件夹。
        \myPicture[0.5\textwidth]{./bash/1.png}{cd 命令}

        使用 ls 命令可以查看当前文件夹下的文件。
        \myPicture[0.5\textwidth]{./bash/2.png}{ls 命令}

        ls 命令后面有一些可选的参数,这里列举几个,更多的可以通过 man ls 查看。
        \begin{itemize}
            \item \texttt{-a} 显示所有文件,包括隐藏文件。
            \item \texttt{-l} 显示文件详细信息
            \item \texttt{-d} 显示目录本身,而不是内容
            \item \texttt{-h} 显示文件大小,通常搭配 \texttt{-l}
            \item \texttt{-F} 在末尾添加字符指示文件类型
            \item \texttt{--color} 使用颜色突出显示文件类型
            \item \texttt{-t} 按照时间排序
        \end{itemize}

        使用 pwd 命令可以查看当前处在那个文件夹下。
        \myPicture[0.5\textwidth]{./bash/3.png}{pwd 命令}

        使用 mkdir 可以创建一个新的文件夹。
        \myPicture[0.5\textwidth]{./bash/4.png}{mkdir 命令}

        使用 rm <NAME> 可以删除一个文件。
        \myPicture[0.5\textwidth]{./bash/9.png}{rm 命令}

        使用 rm -rf <NAME> 可以删除一个文件夹,注意文件夹不能单纯的用 rm 删除。
        \myPicture[0.5\textwidth]{./bash/5.png}{rm rf 命令}

        使用 touch 命令创建一个文件。
        \myPicture[0.5\textwidth]{./bash/6.png}{touch 命令}

        使用 echo 可以向文件后面追加内容。
        \myPicture[0.5\textwidth]{./bash/7.png}{echo 命令}

        使用 vim 去查看结果,也可以使用 cat 命令查看
        \myPicture[0.5\textwidth]{./bash/7.1.png}{查看结果}

        使用 cp 命令可以拷贝一个文件到新位置
        \myPicture[0.5\textwidth]{./bash/8.png}{cp 命令}

        使用 mv 可以移动文件位置,不指定新位置就是重命名
        \myPicture[0.5\textwidth]{./bash/10.png}{mv 命令}

        使用 head 可以查看文件前几行的内容,这里我虽然写了10行但是文件只有1行,tail 命令则是看文档最后几行
        \myPicture[0.3\textwidth]{./bash/11.png}{head 和 tail 命令}

        简单的应用一下 shell 中的 for 循环和 if命令。
        for 之后要跟着 do,以 done 来结尾;
        if 之后也要跟着 do,以 fi 来结尾。
        \myPicture[0.3\textwidth]{./bash/12.png}{for 和 if 命令}


    \section{vim}\label{sec:vim}

        Vim 是一种非常强大的文本编辑器,它采用了不同于传统文本编辑器的操作模式。Vim 主要有三种模式,每种模式都有其特定的功能和用途:

        \begin{enumerate}
            \item \textbf{普通模式}:
            \begin{itemize}
                \item \textbf{功能}:这是 Vim 启动时的默认模式。普通模式用于执行各种编辑命令和导航操作。

                \item \textbf{操作}:按 \texttt{h}、\texttt{j}、\texttt{k}、\texttt{l} 键可以移动光标,
                按 \texttt{d} 可以删除文本,按 \texttt{y} 可以复制文本,按 \texttt{p} 可以粘贴文本。

                \item \textbf{切换}:要进入普通模式,通常只需按 \texttt{Esc} 键。
            \end{itemize}

            \item \textbf{插入模式}:
            \begin{itemize}
                \item \textbf{功能}:插入模式用于直接输入和编辑文本。

                \item \textbf{操作}:要进入插入模式,在普通模式下按 \texttt{i}(在光标前插入),\texttt{I}(在行首插入),
                \texttt{a}(在光标后插入),\texttt{A}(在行尾插入),\texttt{o}(在光标下新建一行),或 \texttt{O}(在光标上方新建一行)。

                \item \textbf{退出}:按 \texttt{Esc} 键可以退出插入模式,返回普通模式。
            \end{itemize}

            \item \textbf{可视模式}:
            \begin{itemize}
                \item \textbf{功能}:可视模式用于选择文本。在这个模式下可以选择一部分文本,进行复制、删除或格式化。

                \item \textbf{操作}:要进入可视模式,在普通模式下按 \texttt{v} 进入字符选择模式,按 \texttt{V} 进入行选择模式,
                按 \texttt{Ctrl+v} 进入块选择模式。选择文本后,可以执行操作,如按 \texttt{d} 删除选择的文本,
                按 \texttt{y} 复制选择的文本,或按 \texttt{p} 粘贴文本。

                \item \textbf{退出}:进入普通模式或插入模式即可。
            \end{itemize}
        \end{enumerate}

        这三种模式让 Vim 的操作非常高效,因为每种模式都有特定的功能和用途,可以根据需要灵活切换。在 Vim 中,掌握这三种模式的使用是提高编辑效率的关键。


        \newpage
        Vim 常用的快捷键,可以相互搭配并结合数字使用
        \begin{itemize}
            \item \texttt{h}:将光标向左移动一个字符。
            \item \texttt{j}:将光标向下移动一行。
            \item \texttt{k}:将光标向上移动一行。
            \item \texttt{l}:将光标向右移动一个字符。
            \item \texttt{dd}:删除当前行。
            \item \texttt{yy}:复制当前行。
            \item \texttt{p}:粘贴复制或剪切的内容到光标后。
            \item \texttt{u}:撤销上一个操作。
            \item \texttt{Ctrl+r}:重做撤销的操作。
            \item \texttt{i}:进入插入模式,在光标前插入文本。
            \item \texttt{Esc}:退出插入模式,返回普通模式。
            \item \texttt{:w}:保存文件。
            \item \texttt{:q}:退出 Vim(如果有未保存的更改,则会提示保存)。
            \item \texttt{:wq}:保存文件并退出 Vim。
        \end{itemize}

        更多的快捷键你可以点击进入\href{https://www.cnblogs.com/markleaf/p/7808817.html}{这篇博客}查看


    \section{感悟体会}\label{sec:}
        在学习了 Shell 和 Vim 之后,我逐渐认识到它们各自的优点与局限。
        Shell 在处理简单的命令和自动化任务时表现出色,但当面临更复杂的需求时,Shell 简单的界面处理起来十分复杂;
        在这种情况下,Python 的优势变得显而易见,Python 的可读性和灵活性使得它在处理复杂数据和算法时更为高效。

        另一方面,Vim 是一款功能强大的编辑器,能够极大地提升我的文本编辑效率,但它的学习曲线确实比较陡峭。
        刚开始使用 Vim 时,我感到非常困惑,其独特的模式和快捷键需要时间去适应,Vim 的三种模式需要灵活的切换使用,才能真正体会到 Vim 的灵魂。

        尽管如此,我也想象到了在处理大型项目时,Vim 肯定有它的局限性。项目规模增大时,单靠 Vim 的功能和操作方式会显得力不从心。
        但是好消息是我发现,现代的集成开发环境(IDE)基本都会提供 Vim 插件,使得我可以在享受 IDE 强大功能的同时,利用 Vim 高效的编辑体验。
        这种结合对二者起到了优势互补的作用,十分好用。

        综上所述,通过对 Shell 和 Vim 的学习,我认识到每种工具都有其特定的优势和适用场景。
        Shell 脚本在系统任务自动化中表现突出,而在处理复杂问题时不如 Python 显得更为得心应手。
        Vim 虽然强大,但在大型项目中结合 IDE 的 Vim 插件使用,可以有效弥补其不足。
        我学会了根据实际需求选择和结合工具,以最大化提升我的工作效率和编程体验。

\end{document}