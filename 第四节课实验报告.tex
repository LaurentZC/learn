\documentclass[16pt]{lzc}

\title{学习实验报告}
\author{刘志才}
\course{系统开发工具基础}
\date{9月13日}

% Document
\begin{document}

    \maketitlepage
    \tableofcontents
    \newpage
    \setcounter{page}{1}

    这里是我的 GitHub 仓库: \href{https://github.com/LaurentZC/learn}{https://github.com/LaurentZC/learn}
    
    \section{调试及性能分析}\label{sec:1}

        \subsection{打印调试法与日志}\label{subsec:2}
            \begin{itemize}
                \item 打印调试法:在发现问题的地方添加打印语句,通过不断重复此过程获取足够的信息以找到问题的根本原因。
                \item 日志:相比于简单的打印语句,日志具有以下优势:
                \begin{itemize}
                    \item 可以将日志写入文件、socket或发送到远端服务器。
                    \item 支持严重等级(如INFO, DEBUG, WARN, ERROR),便于过滤。
                    \item 日志中可能包含足够的信息以帮助定位问题。
                \end{itemize}

            \end{itemize}

        \subsection{调试器}\label{subsec:5}
            当打印调试法无法满足需求时,使用调试器是一个有效的选择。调试器允许开发者与正在执行的程序进行交互,提供以下功能:

            \begin{itemize}
                \item 暂停程序执行。
                \item 单步执行程序。
                \item 查看变量的值。
                \item 设置断点。
            \end{itemize}

            在 IDE 中在锁定的错误区间打上端点
            \myPicture[0.5\textwidth]{./pytorch/img.png}{端点}
            然后通过调试模式逐步调试,一般是这样的按钮
            \myPicture[0.1\textwidth]{./pytorch/img_1.png}{调试按钮}
            \newpage


    \section{元编程}\label{sec:2}
        元编程有构建系统、依赖管理和持续集成三个主要部分。
        他们不仅是软件开发中的重要工具和流程,它们还在团队协作和项目管理中发挥着关键作用。
        构建系统如 make,通过自动化编译和管理依赖,可以极大地提高了开发效率,现代也有更加优秀的 CMake, XMake 等。
        依赖管理和语义版本控制则确保了项目的稳定性,减少了因版本不兼容而导致的问题。
        持续集成的引入,使得代码变更后的测试和部署变得更加高效和可靠,在团队协作中可以快速响应变化。

        如下就是一个简单 CMake 来管理一个 C++ 的项目
        \begin{minted}{cmake}
cmake_minimum_required(VERSION 3.20)

# 要求至少 C++20 标准
set(CMAKE_CXX_STANDARD 20)
set(CMAKE_CXX_STANDARD_REQUIRED ON)

# 定义项目名称变量
set(PROJECT_NAME test)

# 设置项目
project(${PROJECT_NAME} C CXX)

# 设置 Release 模式
if (NOT CMAKE_BUILD_TYPE)
    set(CMAKE_BUILD_TYPE Release)
endif ()

# 查找 fmt 库
find_package(fmt REQUIRED)

# 添加 .cpp 文件
file(GLOB SOURCES src/main.cpp)

# 添加可执行文件
add_executable(${PROJECT_NAME} ${SOURCES})

# 添加头文件
target_include_directories(${PROJECT_NAME} PRIVATE ${CMAKE_SOURCE_DIR}/includes)
# 设置 ncurses 头文件和库路径
target_include_directories(${PROJECT_NAME} PRIVATE "C:/msys64/mingw64/include")
target_link_directories(${PROJECT_NAME} PRIVATE "C:/msys64/mingw64/lib")

# 开启 O2 优化
target_compile_options(${PROJECT_NAME} PRIVATE -O2)

# 删除 Windows 定义的 min 和 max
if (WIN32)
    target_compile_definitions(${PROJECT_NAME} PRIVATE NOMINMAX _USE_MATH_DEFINES)
endif ()

# 链接 fmt 和 Curses 库
target_link_libraries(${PROJECT_NAME} PRIVATE ncursesw fmt::fmt)

# 设置 ccache 加速编译
find_program(CCACHE_PROGRAM ccache)
if (CCACHE_PROGRAM)
    message(STATUS "Found CCache: ${CCACHE_PROGRAM}")
    set_property(GLOBAL PROPERTY RULE_LAUNCH_COMPILE ${CCACHE_PROGRAM})
    set_property(GLOBAL PROPERTY RULE_LAUNCH_LINK ${CCACHE_PROGRAM})
endif ()
        \end{minted}


    \section{大杂烩}\label{sec:3}

        \paragraph{修改键位映射}
            修改键位映射是程序员提高工作效率的一种常见做法。
            通过软件,可以将不常用的按键(如Caps Lock)重新映射为更常用的功能键(如Ctrl或Escape)。
            这种自定义配置可以帮助用户更方便地执行常用操作,提升输入效率。

            当前各种软件都提供了快捷键,按照自己的习惯修改他们可以极大地提高自己的开发效率。

        \paragraph{VPN}
            虚拟私人网络(VPN)是一种通过加密连接保护用户在线隐私的技术。
            使用VPN后,我们的网络流量会通过VPN服务器转发,从而隐藏真实IP地址并加密数据传输。
            VPN可以提高安全性,但仍然需谨慎选择服务提供商,以避免潜在的隐私风险。

            最常见的,我们可以借助VPN去加速一些网站如 GitHub 等。

        \paragraph{Markdown}
            Markdown是一种轻量级的标记语言,旨在使文本格式化变得简单易用。
            它广泛应用于文档编写、博客和在线论坛中。
            通过简单的符号(如*、\#等),用户可以快速创建标题、列表、链接和代码块等格式,使得文档更加清晰易读。

            Markdown 更加适合电子阅读,它页面从头到尾连续一致,没有 word 和 \LaTeX 这种带来的页面割裂感。
            更是可以通过 mkdos 等工具把自己的 Markdown 部署到网页,通过 GitHub Pages 可以写一些简单的``书''。

            Markdown 简单的格式更是适合哪里做笔记。如\href{run:./实例.md}{实例(点击打开)}中就展示了本人做过的笔记。

        \paragraph{开机引导}
            开机引导是计算机启动时加载操作系统的过程。
            BIOS或UEFI负责初始化硬件并选择启动设备。
            用户可以通过特定的键进入设置界面,配置启动顺序或选择从USB等外部设备启动。
            这一过程对于安装新操作系统或进行系统修复非常重要。

            GRUB 是一个开源的引导加载程序,主要用于启动 Linux 系统,但也能引导其他操作系统,如 Windows。
            它允许用户在启动时选择要进入的操作系统,支持多系统引导。GRUB 具有灵活性,能处理各种文件系统和引导模式,包括 UEFI 和传统 BIOS 系统。
            通过其配置文件和自动更新工具,GRUB 能够适应不同的硬件配置和操作系统需求,使得多操作系统环境的管理更加简便。
            利用 GRUB 我们可以使用 Windows 和 Linux 的双系统,非常地舒服。

        \paragraph{GitHub}
            在第一节课已经学习了 GitHub 的一些简单应用,在本节课又学习了 GitHub 的 fork / PR 等功能。

            在本节课中,我们学习了 GitHub 的两个重要功能:fork 和 pull request(PR)。

            \begin{itemize}
                \item Fork: 将某个仓库的完整副本复制到自己的账户下,从而允许在独立的环境中进行修改而不会影响原始仓库。
                \item Pull Request(PR): 一种用于提交更改的请求,允许你将自己在 fork 中做的更改提议合并到原始仓库。
            \end{itemize}

            结合使用 fork 和 PR,可以有效地参与开源项目的开发和协作,
            我们先 fork 一份到自己的账户下,clone 下来之后在本地进行修改,然后提交一个 PR,
            在维护者审查无误后可以进行合并,从而完成一次对开源项目的贡献,
            同时可以确保代码的质量和项目的进步。


    \section{pytorch}\label{sec:pytorch}

    PyTorch 是一个开源的深度学习框架,由 Facebook 的人工智能研究团队开发。
    它提供了灵活的张量计算和自动求导功能,使得构建和训练神经网络变得简单直观。
    PyTorch 的动态计算图特性允许用户在运行时修改模型结构,使得调试和实验变得更加方便。
    此外,PyTorch 还支持使用 CUDA 进行 GPU 加速,从而提高计算效率。

    \paragraph{生成张量数据}
        \begin{minted}{Python}
n_data = torch.ones(100, 2)                   # 创建一个 100 * 2 的元素全为 1 的矩阵
class_0_data = torch.normal(2 * n_data, 1)    # 生成均值为 2 * n_data,标准差为 1 的正态分布数据
class_0_labels = torch.zeros(100)             # 标签0,100个样本
class_1_data = torch.normal(-2 * n_data, 1)   # 生成均值为 -2 * n_data,标准差为 1 的正态分布数据
class_1_labels = torch.ones(100)              # 标签1,100个样本
        \end{minted}

    \paragraph{数据转换}
        \begin{minted}{Python}
# 合并 class_0_data 和 class_1_data,类型转换为 FloatTensor
x = torch.cat((class_0_data, class_1_data), 0).type(torch.FloatTensor)
# 合并 class_0_labels 和 class_1_labels,类型转换为 LongTensor
y = torch.cat((class_0_labels, class_1_labels), 0).type(torch.LongTensor)
# 将数据和标签包装为 Variable
x, y = Variable(x), Variable(y)
        \end{minted}
        第一句代码将 class\_0\_data 和 class\_1\_data 沿第一个维度(行)拼接在一起,形成一个包含所有样本的数据张量,并将其数据类型转换为 FloatTensor。
        第二句代码将 class\_0\_labels 和 class\_1\_labels 沿第一个维度拼接在一起,形成一个包含所有标签的张量,并将其数据类型转换为 LongTensor。
        最后将数据张量 x 和标签张量 y 包装为 Variable 对象,以便在需要进行梯度计算时使用。

    \paragraph{定义神经网络}
        \begin{minted}{Python}
class SimpleNN(torch.nn.Module):
    def __init__(self, n_feature, n_hidden, n_output):
        super(SimpleNN, self).__init__()
        self.hidden = torch.nn.Linear(n_feature, n_hidden)  # 隐藏层,输入特征数到隐藏层神经元数
        self.out = torch.nn.Linear(n_hidden, n_output)  # 输出层,隐藏层神经元数到输出类别数

    def forward(self, x):
        x = F.relu(self.hidden(x))  # 对隐藏层输出应用 ReLU 激活函数
        x = self.out(x)  # 通过输出层
        return x
        \end{minted}

        这段代码定义了一个简单的神经网络 SimpleNN:

        \_\_init\_\_ 方法初始化了网络结构:一个隐藏层和一个输出层。
        隐藏层接受 n\_feature 个输入特征,输出 n\_hidden 个神经元;
        输出层接受 n\_hidden 个输入,输出 n\_output 个类别。

        forward 方法定义了前向传播过程:首先通过隐藏层,并使用 ReLU 激活函数,然后将结果传递到输出层,得到最终地输出。

    \paragraph{定义优化器和损失函数}
        \begin{minted}{Python}
optimizer = torch.optim.SGD(net.parameters(), lr = 0.02)  # 使用随机梯度下降优化器
loss_func = torch.nn.CrossEntropyLoss()                   # 交叉熵损失函数,用于多分类问题
        \end{minted}

    \paragraph{模型训练和可视化输出}
        \begin{minted}{Python}
for t in range(100):  # 训练 100 次
    out = net(x)  # 前向传播:通过网络计算输出
    loss = loss_func(out, y)  # 计算损失

    optimizer.zero_grad()  # 清空梯度
    loss.backward()  # 反向传播:计算梯度
    optimizer.step()  # 更新参数

    if t % 5 == 0:  # 每 5 次训练迭代更新一次图
        plt.cla()  # 清除当前图形
        prediction = torch.max(out, 1)[1]  # 预测输出的类别
        plt.scatter(x.data.numpy()[:, 0], x.data.numpy()[:, 1],
                    c = prediction.data.numpy(),
                    s = 100, lw = 0, cmap = 'RdYlGn')  # 绘制散点图
        correct = torch.sum(torch.eq(prediction.data, y.data)).item()  # 计算正确预测的数量
        accuracy = correct / y.data.size(0)  # 计算准确度
        plt.text(1.5, -4, 'Accuracy=%.2f' % accuracy,
                fontdict = {'size': 20, 'color': 'red'})  # 显示准确度
        plt.pause(0.1)  # 暂停,以便动态显示图形

plt.ioff()  # 关闭交互模式
plt.show()  # 显示最终图形
        \end{minted}

        这段代码展示了训练模型的过程,同时利用上节课学习的 matplotlib 库进行可视化的输出,方便我们查看结果。

    \paragraph{查看结果}
        这里展现了结果
        \myPicture[0.3\textwidth]{./pytorch/0.png}{初始状态}
        \myPicture[0.3\textwidth]{./pytorch/1.png}{5次训练后}
        \myPicture[0.3\textwidth]{./pytorch/2.png}{100次训练后}

        \newpage
        这是一段分类的神经网络,下面这段代码展示一段数据拟合的神经网络


        \begin{minted}{Python}
# 生成合成数据
# 创建一个包含100个点的列向量,范围从-1到1
inputs = torch.unsqueeze(torch.linspace(-1, 1, 100), dim = 1)
# 使用二次关系生成目标值,并添加一些无关的噪声
targets = inputs.pow(2) + 0.2 * torch.rand(inputs.size())

# 绘制合成数据
plt.scatter(inputs.data.numpy(), targets.data.numpy())  # 绘制inputs与targets的散点图
plt.show()  # 显示图像


# 定义一个简单的神经网络模型
class SimpleNet(nn.Module):
    def __init__(self, input_size, hidden_size, output_size):
        super(SimpleNet, self).__init__()
        self.hidden_layer = nn.Linear(input_size, hidden_size)  # 隐藏层
        self.output_layer = nn.Linear(hidden_size, output_size)  # 输出层

    def forward(self, x):
        x = F.relu(self.hidden_layer(x))  # 应用ReLU激活函数
        x = self.output_layer(x)  # 通过输出层
        return x


# 初始化神经网络
model = SimpleNet(1, 10, 1)  # 1个输入特征,10个隐藏单元,1个输出特征
print(model)  # 打印网络架构

plt.ion()  # 开启交互模式以便实时更新
plt.show()  # 显示图像窗口

# 设置优化器和损失函数
optimizer = torch.optim.SGD(model.parameters(), lr = 0.5)  # 使用随机梯度下降优化器
loss_function = nn.MSELoss()  # 使用均方误差损失函数

# 训练循环
for epoch in range(100):
    predictions = model(inputs)  # 前向传播

    loss = loss_function(predictions, targets)  # 计算损失

    optimizer.zero_grad()  # 清除梯度
    loss.backward()  # 反向传播以计算梯度
    optimizer.step()  # 更新权重

    if epoch % 5 == 0:  # 每5步,更新一次图像
        plt.cla()  # 清除当前图像
        plt.scatter(inputs.data.numpy(), targets.data.numpy())  # 绘制原始数据
        plt.plot(inputs.data.numpy(), predictions.data.numpy(),
                'r-', lw = 5)  # 绘制模型的预测结果
        plt.text(0.5, 0, 'Loss = %.4f' % loss.item(),
                fontdict = {'size': 20, 'color': 'red'})  # 显示当前损失
        plt.pause(0.5)  # 暂停以更新图像

plt.ioff()  # 关闭交互模式
plt.show()  # 显示最终图像
        \end{minted}

        这里展现了结果
        \myPicture[0.3\textwidth]{./pytorch/3.png}{初始状态}
        \myPicture[0.3\textwidth]{./pytorch/4.png}{5次训练后}
        \myPicture[0.3\textwidth]{./pytorch/5.png}{100次训练后}


    \section{感悟体会}\label{sec:5}
    本节课学习了较多的内容,包括调试与性能分析、元编程、CMake、键位映射、VPN、Markdown、开机引导,以及 PyTorch 等。

    首先,调试和性能分析是任何编程和系统开发中的核心技能。
    通过调试,我学会了如何有效地识别和解决代码中的问题,从而提高了开发效率。
    IDE 提供的静态分析和端点调试功能也让我们的调试变得十分方便。

    在元编程的学习中,我体会到了如何利用工具和技术来简化复杂的构建过程,我尤其对 CMake 感兴趣。
    CMake 作为一个跨平台的构建系统生成工具,使得在不同平台和环境中构建项目变得更加一致和高效。
    通过学习 CMake 的用法,我能够更好地管理项目的构建过程,并处理依赖关系。
    我现在用它去管理了我的 C++ 项目,十分地方便,可以很容易地让我在 Windows 和 linux 之间进行移植。

    修改键位映射的经历让我更加灵活地适应了不同的开发环境。
    无论是在 Windows 还是 Linux 上,自定义键位映射可以保证键位永远是我熟悉的,避免熟悉不同的快捷键,让流程更加顺畅。

    在学习 Markdown 的过程中,我发现它是编写文档和笔记的利器。
    Markdown 简洁的语法和强大的功能让我能够快速创建格式良好的文档,提升了我的文档编写效率。

    开机引导的配置让我了解了计算机系统启动的过程,这对于系统级别的调试和维护非常重要。
    对于我这种双系统用户帮助十分巨大。

    最后我用 PyTorch 实现了两个简单的神经网络。
    我分别构建一个简单的分类网络和曲线拟合模型,让我对深度学习的原理和实践有了初步了解。
    PyTorch 强大的库和灵活的框架使得模型的训练和调优变得更加直观,也让我体会到了从数据准备到模型训练和评估的完整流程。
    不得不感慨现在的 AI 学习效率是真地高,但是我也认识到了只有大量的数据和不断地试错才能提高 AI 的准确度。

    总体来说,本节课学习的知识不仅丰富了我的技术知识,也提升了我的解决问题能力。
    每一项技能和每一种工具的掌握都方便了我的学习过程,也让我对未来的技术挑战充满了信心。
    我期待将这些知识应用到更复杂的项目中,继续提升自己的技能和经验。
\end{document}