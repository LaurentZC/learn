    \subsection{任务控制}\label{subsec:5}

        \paragraph{暂停任务}\label{par:暂停}
            通常情况下我们可以直接使用 <C-c>\footnote{我们使用 <C-key> 来表示按下 Ctrl + key} 就可以终止一个任务,例如
            \myPicture[0.3\textwidth]{./Shell/1.png}{<C-c> 终止程序}
            但是有的时候 <C-c> 不能终止程序,因为这些程序可能 可能会捕捉或忽略 SIGINT 信号,
            或者他们在后台运行(即加上了 \& 符号),<C-c> 只会影响当前前台进程。

        \paragraph{kill 命令}
            SIGTERM 则是一个更加通用的、也更加优雅地退出信号。
            为了发出这个信号我们需要使用 kill 命令, 它的语法是: kill -TERM <PID>。
            通常情况下,kill 是默认发出 SIGTERM 的,因此可以省略为 kill <PID>。
            这里的 <PID> 是进程的标识符,你可以使用
            \begin{minted}{Shell}
ps aux
            \end{minted}
            列出所有的进程,查看他们的 PID;
            或者你只关心某一个进程,则可以使用
            \begin{minted}{Shell}
ps aux | grep <process_name>
            \end{minted}
            这样,我们就可以优雅地终止进程,并在必要时强制停止它。

        \paragraph{其他信号}
            \begin{itemize}
                \item <C-z> 可以发出一个 SIGTSTP 信号, 会让进程暂停。
                \myPicture[0.3\textwidth]{./ml/1.png}{<C-z> 停止进程}

                \item 使用 nohup 来创建一个忽略 SIGHUP 的进程,使用 \& 让他在后台进行。

                使用 jobs 命令来查看所有进程。
                \myPicture[0.3\textwidth]{./ml/2.png}{nohup 和 jobs}

                \item 使用 bg 命令来让暂停的进程重新工作
                \myPicture[0.3\textwidth]{./ml/3.png}{bg 命令}

                \item 使用 kill -STOP 来暂停进程
                \myPicture[0.3\textwidth]{./ml/4.png}{kill -STOP}

                \item 尝试使用 kill -SIGHUP 去暂停进程2
                \myPicture[0.3\textwidth]{./ml/5.png}{kill -SIGHUP}
                我们使用 jobs 查看后发现进程2并没有暂停,这是因为我们在\hyperref[par:暂停]{上面}提到的
                nohup 忽略了 SIGHUP

                \item 在\hyperref[fig:./ml/5.png]{kill -SIGHUP}最后两行,我们使用了 kill 来结束进程2
                SIGKILL 是一个特殊的信号,它不能被进程捕获,并且它会马上结束该进程。
                不过这样做会有一些副作用,例如留下孤儿进程。
            \end{itemize}

\subsection{tmux}\label{subsec:tmux}

    tmux 是一个终端多路复用器,它可以允许我们基于面板和标签分割出多个终端窗口,
    这样我们就可以同时与多个 shell 会话进行交互。

    \subsubsection{会话}
        \begin{minted}[bgcolor=lightgray, frame=single]{Shell}
tmux                # 开启一个新的对话
tmux new -S NAME    # 以 NAME 名称开启一个新的会话
tmux ls             # 列出所有会话
<C-b> + d           # 在 tmux 会话中键入,会分离当前会话
        \end{minted}
        \myPicture[0.5\textwidth]{./tmux/1.png}{创建会话}

        \begin{minted}[bgcolor=lightgray, frame=single]{Shell}
tmux a              # 重新连接最后一个会话
tmux a -t NAME      # 指定链接某个会话
        \end{minted}
        \myPicture[0.3\textwidth]{./tmux/2.png}{重新链接会话}
        \myPicture[0.3\textwidth]{./tmux/3.png}{指定 Hello 链接}

    \subsubsection{窗口}
        \begin{minted}[bgcolor=lightgray, frame=single]{Shell}
<C-b> + c       # 创建一个新的窗口,<C-d> 关闭
<C-b> + N       # 跳转到第N个窗口
<C-b> + p       # 切换到前一个窗口
<C-b> + n       # 切换到后一个窗口
<C-b> + ,       # 重命名当前窗口
<C-b> + w       # 列出当前所有窗口
        \end{minted}
        \myPicture[0.3\textwidth]{./tmux/4.png}{窗口}

    \subsubsection{面板}

        \begin{minted}[bgcolor=lightgray, frame=single]{text}
<C-b> + "        # 水平分割
<C-b> + %        # 垂直分割
<C-b> + <方向>    # 切换到指定方向的面板,<方向> 指的是键盘上的方向键
<C-b> + z        # 切换当前面板的缩放
<C-b> + [        # 开始往回卷动屏幕。可以按下空格键来开始选择,回车键复制选中的部分
<C-b> + <空格>    # 在不同的面板排布间切换
        \end{minted}

\subsection{SSH}\label{subsec:ssh}
    SSH 是一种网络协议,用于在不安全的网络中安全地访问和管理远程计算机。
    它提供了一种加密的方式来连接到远程计算机,从而确保数据的机密性和完整性。
    SSH 广泛用于远程登录、远程命令执行和文件传输等任务。

    这里我们以配置 SSH 链接 GitHub 为例,我们可以使用
    \begin{minted}{Shell}
ssh-keygen -t rsa -b 4096
    \end{minted}
    来创建一个 SSH 的密钥,其中 -t rsa 指定要生成的密钥类型为 RSA,-b 4096 指定密钥的位数(长度)。
    \myPicture[0.3\textwidth]{./ssh/1.png}{生成密钥}

    创建完成,我们去查看一下
    \myPicture[0.3\textwidth]{./ssh/2.png}{ssh 目录}
    我们发现,有一个 id\_rsa 和一个 id\_rsa.pub。

    其中 id\_rsa 是自己的私钥,不应提供给任何人;id\_rsa.pub 是公钥。
    我们把公钥提交到 GitHub,通过公钥和私钥的配对,就可以来链接到 GitHub

    当然我们需要先做一些配置,创建一个 config 文件,然后在里面写入 GitHub 的服务器。
    \myPicture[0.3\textwidth]{./ssh/4.png}{创建 config 文件}
    \myPicture[0.3\textwidth]{./ssh/3.png}{写入配置}