\documentclass{lzc}

\title{git和\LaTeX{}学习实验报告}
\author{刘志才}
\course{系统开发工具基础}
\date{8/23}

\setLanguage{bash}

\begin{document}
    \maketitlepage
    \tableofcontents
    \newpage
    \setcounter{page}{1}


    \section{写在正文之前的话}\label{sec:1}
        这是系统开发工具基础 8月23日 第一节课的实验报告,在附件中放了本文档和自制模板的源码。

        其中 git 命令这地方,由于本人还不会在 \LaTeX 里面调整图片,所以放在了附件,这里只列出了自学的指令。

        \LaTeX 本人学过,所以只写了一些简单的代码。

        GitHub 地址:\href{https://github.com/LaurentZC/learn}{https://github.com/LaurentZC/learn}


    \section{git 的学习}\label{sec:git}

        \subsection{git 的初始化}\label{subsec:1.1}
            使用 git 之前先来配置一下,首先是三个查看的命令

            \begin{lstlisting}[label={lst:1}]
        git --version               # 查看 git 的版本,通常用来确定是否安装了 git
        git remote --version        # 查看远程仓库,和 GitHub 配合使用
        git config --global -list   # 查看用户的信息
            \end{lstlisting}

            接下来设置 git 的用户信息
            \begin{lstlisting}[label={lst:2}]
        git config --global user.name <name>            # 设置用户名
        git config --global user.email <email>          # 设置用户邮箱
        git config --global credential.helper store     # 将用户信息存起来,以免需要重复输入
            \end{lstlisting}

        \subsection{git 仓库的创立和删除}\label{subsec:1.2}
            \begin{lstlisting}[label={lst:3}]
        git init <name>         # 初始化一个本地仓库
        git clone <address>     # 克隆一个远程仓库
        git status              # 查看仓库状态:工作区,暂存区
        \rm -rf .git            # 只要把 .git 这个文件删除即可,注意是隐藏文件
            \end{lstlisting}

        \subsection{使用 git 管理文件}\label{subsec:1.3}
            git 分成三部分组成:工作区,暂存区, 版本库。
            \begin{itemize}
                \item 工作区:电脑文件系统中能看到的项目目录
                \item 暂存区:是用于暂存工作区中的改动,但这些改动还没有提交到本地仓库。
                \item 版本库:包含项目的所有版本历史记录,分为本地仓和远程仓。
            \end{itemize}

            \begin{lstlisting}[label={lst:4}]
        git remote add <name> <address>                     # 链接本地仓和远程仓
        git add <file>                                      # 工作区 -> 暂存区,可以使用通配符
        git rm --cached <file>                              # 暂存区 -> 工作区
        git commit -m <message>                             # 暂存区 -> 本地仓
        git push <remote> <local branch>:<remote branch>    # 本地仓 -> 远程仓
        git pull <local> <remote branch>:<local branch>     # 远程仓 -> 本地仓
            \end{lstlisting}

        \subsection{git 版本控制}\label{subsec:1.4}
            每次提交都会有一个唯一的版本号,可以通过 git log 查询

            git 也提供了一个简单方法,HEAD 就表示当前版本,HEAD\textasciicircum 或 HEAD\textasciitilde 表示上一个版本,
            在 \textasciicircum 或 \textasciitilde 后面可以加数字。

            \begin{lstlisting}[label={lst:5}]
        git log
        # 显示提交记录,--oneline是简化版本

        git diff <version> <file>
        # 查看版本之间某个文件的差异
        # 两个版本号对比,一个版本号默认和仓库比,无版本号对比工作区和暂存区
        # 无文件默认比较全部

        git reset <version>
        # 回退版本,参数有 --soft(全部保留),--mixed(保留工作区,默认),--hard(不保留)

        git reflog <version>
        # 误操作可以回溯回来
            \end{lstlisting}


    \section{\LaTeX{}的学习使用}\label{sec:latex}

        \subsection{\LaTeX{}文件必需品}\label{subsec:2.1}
            在最开始要使用指明文档类型,[]是可选参数,\{\}是类型,主要有四大类型:article,book,report,beamer。
            为了支持中文,可以使用:ctexart,ctexbook,ctexrep,ctexbeamer。
            \begin{verbatim}
        \documentclass[]{} % 指明文档类型
        % 导言区,可以写命令

        \begin{document}
        % 正文
        \end{document}
            \end{verbatim}

            下面均以使用较多的 article/ctexart 为例

        \subsection{相关命令}\label{subsec:2.2}
            设置\LaTeX 文章标题等
            \begin{verbatim}
        \title{}  % 标题
        \author{} % 作者
        \date{}   % 创作日期
        \today    % 今天的日期

        \maketitle          % 正文使用,打印标题,作者,创作日期
        \tableofcontents    % 生成目录
            \end{verbatim}


            设置文章的各级标题,以 “sub” 分级
            \begin{verbatim}
        \section{}          # 一级标题
        \subsection{}       # 二级标题
        \subsubsection{}    # 三级标题
            \end{verbatim}

            宏包的使用:宏包是用来扩展 \LaTeX 的功能,通过安装不同的宏包可以实现一些复杂排版功能,例如插入复杂的列表表格、插入公式和特殊符号、插入代码、设置文档版式等。
            宏包的使用很简单只需要在导言区加入
            \begin{verbatim}
        \usepackage[]{}
            \end{verbatim}
            即可使用。
            下面列举一下常用的宏包

            \begin{tabular}{|c|c|}
                \hline
                \text{宏包名} & \text{说明}               \\
                \hline
                amsmath    & AMS \text{数学公式}         \\
                \hline
                geometry   & \text{修改页面尺寸,页边距,页眉页脚等} \\
                \hline
                titlesec   & \text{修改标题格式}           \\
                \hline
            \end{tabular}

        \subsection{列表和表格}\label{subsec:2.3}
            \begin{verbatim}
        \begin{itemize} % 无序列表
            \item 第一项
            \item 第二项
            \item 第三项
        \end{itemize}

        \begin{enumerate} % 有序列表
            \item 第一项
            \item 第二项
            \item 第三项
        \end{enumerate}

        \begin{tabular}{|c|c|c|} % 表格,l,c,r分别表示左中右对齐
            \hline
            列1 & 列2 & 列3 \\
            \hline
            数据1 & 数据2 & 数据3 \\
            \hline
            数据4 & 数据5 & 数据6 \\
            \hline
        \end{tabular}
            \end{verbatim}

        \subsection{数学公式}\label{subsec:2.4}
            \LaTeX 在排版数学公式上有着极大的优势,\LaTeX 提供了三种方式:行内公式,行间公式,数学环境

            行内公式例如 $e^{i\pi} + 1 = 0$,只需要用一对\$包裹起来
            \begin{verbatim}
        $e^{i\pi} + 1 = 0$
            \end{verbatim}

            行内公式例如 \[ e^{i\pi} + 1 = 0 \] 可以用一对\$\$包裹,也可以使用\textbackslash$[$ \textbackslash$]$ 包裹(推荐)
            \begin{verbatim}
        \[
        e^{i\pi} + 1 = 0
        \]
            \end{verbatim}

            \LaTeX 提供的数学环境很多,这里举一个多行公式的例子,数学环境中的公式是自动标号的(也有不标号的),可以方便我们引用而不是重复写公式
            \begin{align}
                e^{i\pi} + 1 &= 0 \\
                E &= mc^{2}
            \end{align}

            \begin{verbatim}
        \begin{align}
            e^{i\pi} + 1 &= 0 \\
            E &= mc^{2}
        \end{align}
            \end{verbatim}
            \newpage


    \section{感悟体会}\label{sec:3}
        本节课,我学习和使用了 Git 以及\LaTeX,我充分感受到了这两者在提高工作效率和文档管理方面的便利性。

        Git 是一个强大的版本控制系统,它用户追踪文件的每一次更改,回溯到任何一个历史版本,尤其在开发时还支持多人协作而不会产生冲突。
        Git 的分支管理功能也十分强大,但很可惜由于时间关系,我暂时未学习到这里,在课下我会继续学习。
        此外,Git 的远程仓库功能(GitHub)也使得代码分享和版本控制变得非常简单,这些极大的方便了其他课程小组合作开发程序。
        在学习 Git 的时候,我也遇到了一些困难:在本地仓库向远程仓库推送时 ssh 密钥不匹配,通过b站的相关视频,历经千辛万苦终于解决了。

        \LaTeX 是一种专业的排版系统,特别适合于需要复杂布局和数学公式的文档。
        \LaTeX 允许用户通过编写简单的标记语言来定义文档的结构和格式,而最终生成的文档则具有一致且专业的排版效果。
        这样可以让用户专注于内容本身,而不需要分神反复调整格式和排版。
        \LaTeX 还可以自动生成目录、参考文献和索引,这减少了重复劳动,不需要手动一个个添加或删除。

        总的来说,Git和\LaTeX 这两种工具在工作中都有十分明显的优势。 熟练使用它们,不仅提升能工作效率,还可以优化项目管理和文档编排的流程。
        本节课学习的这两种工具让我受益匪浅,在未来我会多尝试使用这两种工具,以方便管理。
\end{document}
