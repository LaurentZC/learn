        Python 同样是一门高级语言,我们按照一般地学习顺序来学习

        \subsection{基本功能}\label{subsec:}
            首先贴出一份归并排序的代码,然后解释,首先我们要导入一个数学库

            \begin{minted}{Python}
from math import sqrt
            \end{minted}

            导入库有两种方法,一种是像上面写的一样,这个意思是从 math 库中引入 sqrt 这个函数;
            另一种方法是 import math,这样是把 math 库里面的所有函数都引入进来,不过在调用是需要 math.sqrt() 来调用。

            \begin{minted}{Python}
def merge(lst, left, mid, right):
    left_lst = lst[left:mid + 1]
    right_lst = lst[mid + 1:right + 1]

    left_idx = right_idx = 0
    sorted_idx = left

    while left_idx < len(left_lst) and right_idx < len(right_lst):
        if left_lst[left_idx] <= right_lst[right_idx]:
            lst[sorted_idx] = left_lst[left_idx]
            left_idx += 1
        else:
            lst[sorted_idx] = right_lst[right_idx]
            right_idx += 1
        sorted_idx += 1

    while left_idx < len(left_lst):
        lst[sorted_idx] = left_lst[left_idx]
        left_idx += 1
        sorted_idx += 1

    while right_idx < len(right_lst):
        lst[sorted_idx] = right_lst[right_idx]
        right_idx += 1
        sorted_idx += 1
            \end{minted}

            我们使用了 def 来定义一个函数,这个函数接受四个参数,分别是一个数组,左索引,中间索引,右索引;

            数字的这个操作:[mid + 1:right + 1],是对数组进行切片,Python 的数组切片功能十分强大,
            支持正着切片,反着切片,跳着切片(你只需要在用一个:指定步长即可)。

            while 是 Python 的循环结构的关键词,我们也可以使用 for i in range\ldots 来使用 for 循环,
            在 Python 中 for 循环通常比 while 更加高效,不过我们更推荐注重代码的可读性,
            毕竟如果你注重哪一点效率为什么不去用 C++ 呢?

            if 和 else 就是 Python 的分支结构关键词,通常与他们搭配的就是逻辑运算符,
            在 Python 中,你可以使用以下三个关键字:and or not。
            当然 Python 还提供了 elif 关键词。

            \begin{minted}{Python}
def merge_sort(array, left_index, right_index):
    if left_index < right_index:
        mid_index = (left_index + right_index) // 2
        merge_sort(array, left_index, mid_index)
        merge_sort(array, mid_index + 1, right_index)
        merge(array, left_index, mid_index, right_index)
            \end{minted}

            这里呢我们又定义了一个函数,并递归的处理数组进行排序。
            Python 用 // 来表示整除运算符,它返回两个数相除后的整数部分,类似的还有 ** 代表乘方。

            \begin{minted}{Python}
if __name__ == "__main__":
    sample_array = [64, 34, 25, 12, 22, 11, 90]
    print("Original array:", sample_array)
    merge_sort(sample_array, 0, len(sample_array) - 1)
    print("Sorted array:", sample_array)
            \end{minted}

            这里是 Python 的主函数了。
            if \_\_name\_\_ == ``\_\_main\_\_'':
            是 Python 中用来检查一个模块是否作为主程序运行的条件。
            其作用是确保代码块仅在脚本被直接执行时运行,而在脚本作为模块导入时不执行。
            你可以不使用他,但是使用它是一种良好的编程喜欢,推荐。

            你可以使用一个简单的[]来定义一个数组,用 len 求出长度,十分地方便。

            print 是 Python 的打印函数,比 C++ 提供了更强大地格式化输出功能,等会我们就可以看到他。

            \begin{minted}{Python}
    sample_array.append(70)
    sample_array.insert(1, 80)

    print("Array after insert:", sample_array)
    merge_sort(sample_array, 0, len(sample_array) - 1)
    print("Sorted array:", sample_array)
            \end{minted}

            这里呢,我们使用了 append 和 insert 两种方法往数组里面插入元素。
            append 是在数组末尾追加,而 insert 可以让你选择插入位置。

            \begin{minted}{Python}
    sqrt_array = [sqrt(x) for x in sample_array[1:4]]
    print(f"Array after sqrt: {[f'{x:.3f}' for x in sqrt_array]}")
            \end{minted}

            这里我们对数组进行了一个切片,并且对切片的数组进行了开方操作,然后保留小数点后三位打印出来。

            \newpage
            我们在举几个 Python 格式化输出的例子

            \begin{minted}{Python}
    name = "Alice"
    age = 30
    height = 5.6

    # 和C语言挺像的
    print("Name: %s, Age: %d, Height: %.1f" % (name, age, height))
    # 用 format 可以方便的格式化一个字符串
    print("Name: {}, Age: {}, Height: {:.2f}".format(name, age, height))
    # 用 f-string 来格式化
    print(f"Name: {name}, Age: {age}, Height: ${height:,.2f}")
    # 你可以自定义分隔符和结束符
    print("Apple", "Banana", "Cherry", sep=" | ", end=".\n")
            \end{minted}

        \subsection{字典}\label{subsec:3}
            贴代码,解释已经写在了注释
            \begin{minted}{Python}
    # 创建一个字典
my_dict = {'name': 'Charon', 'age': 20, 'city': 'Qing Dao', 'email': 'hello@gmail.com', 'null': 1}

# 打印一下
print("Original Dictionary: ", my_dict)

# 删除一个键值对
del my_dict['null']
# 弹出
city = my_dict.pop('city')
print("Dictionary after delete: ", my_dict)
print("city: ", city)

# 获取一个值,使用 get 方法,如果键不存在则返回默认值
email = my_dict.get('email', 'Not Found')
print("email: ", email)

# 添加一个新的键值对
my_dict['sex'] = 'male'
print("Dictionary after update: ", my_dict)

# 获取所有键
keys = my_dict.keys()
print("Keys:", keys)

# 获取所有值
values = my_dict.values()
print("Values:", values)

# 获取所有键值对
items = my_dict.items()
print("Items:", items)

# 清空字典
my_dict.clear()

# 输出结果
print("Dictionary after clear:", my_dict)
            \end{minted}

        \subsection{元组}\label{subsec:4}
            \begin{minted}{Python}
    # 创建一个元组
my_tuple = (1, 2, 2, 3, 4, 5)

# 计算元组长度
length = len(my_tuple)
print(f"Length: {length}")

# 查找元素的位置
index = my_tuple.index(3)  # 查找元素3的位置
print(f"Index of 3: {index}")

# 计数某个元素出现的次数
count = my_tuple.count(2)
print(f"Count of 2: {count}")

# 访问元组元素
element = my_tuple[2]  # 访问第三个元素
print(f"Element at index 2: {element}")

# 切片操作
slice_tuple = my_tuple[1:4]
print(f"Slice from index 1 to 3: {slice_tuple}")

# 拼接元组
new_tuple = my_tuple + (6, 7)
print(f"Concatenated tuple: {new_tuple}")

# 复制元组
copied_tuple = my_tuple * 2
print(f"Copied tuple: {copied_tuple}")
            \end{minted}